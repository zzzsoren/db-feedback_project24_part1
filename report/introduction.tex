\section{Introduction}
The goal of this project is get hands-on experience setting up a database in PostgreSQL and
looking up data in it. Furthermore the purpose is about us getting feedback as early in the course as possible. 
The report reflects and documents the work carried out by us \textit{(Denis and Søren)} in the project.

\subsection{Formulation of the problem}
Create a database and insert data, then query data.

\subsection{Requirements}
The database should be deployed in PostgreSQL with appropriate commands
to create the tables, by taking into account and enforcing all the integrity constants. About 5-10 records in each table.
\bigbreak
The schema to be implemented is as follows:\\
\begin{itemize}
    \item \texttt{PRODUCT(ProductID, CategoryID, ProductName, Description)}\\
    Information about a product. A product is assoiciated only with a single category
    to fullfill the assignment decription.
    \item \texttt{PRODUCT\_CATEGORY(CategoryID, Name, Description)}\\
    The category is the type of the product.
    \item \texttt{SUPPLIER(SupplierVAT, SupplierName, Address, Phone, Email)}\\
    A supplier supplies the products to a retail store.
    \item \texttt{SUPPLY(InvoiceID, SupplierVAT, Date)}\\
    Relationship between the supplier, the supplied products and the date og supplying.
    \item \texttt{PRODUCT\_SUPPLY(InvoiceID, ProductID, Quantity, Value)}\\
    Product that has been supplied by the supplier.
    The price is what the store pay the supplier pr. unit.
    \item \texttt{SALE(SaleID, Date)}\\
    A sale is between retail and their customers. We interpret this as 
    a sale that could be any combination of the products in stock.
    \item \texttt{SALE\_OF\_PRODUCT(SaleID, ProductID, Quantity, Value)}\\
    The price is what customer pay pr. unit of a particular product.
    \item \texttt{PRODUCT\_RETURN(SaleID, ProductID, Date, Quantity)}\\
    A costumer can return the products to the store that they've bought if they're unhappy.
    \item \texttt{STOCK(ProductID, Quantity)}\\
    The stock is the total 
\end{itemize}



\subsection{Scope}
There is not much to this project. Just create the database, and discuss the methods of retrieving the data
the assignment requires.

